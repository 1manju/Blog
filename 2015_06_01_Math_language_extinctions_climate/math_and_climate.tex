\documentclass[]{article}
\usepackage{lmodern}
\usepackage{amssymb,amsmath}
\usepackage{ifxetex,ifluatex}
\usepackage{fixltx2e} % provides \textsubscript
\ifnum 0\ifxetex 1\fi\ifluatex 1\fi=0 % if pdftex
  \usepackage[T1]{fontenc}
  \usepackage[utf8]{inputenc}
\else % if luatex or xelatex
  \ifxetex
    \usepackage{mathspec}
    \usepackage{xltxtra,xunicode}
  \else
    \usepackage{fontspec}
  \fi
  \defaultfontfeatures{Mapping=tex-text,Scale=MatchLowercase}
  \newcommand{\euro}{€}
\fi
% use upquote if available, for straight quotes in verbatim environments
\IfFileExists{upquote.sty}{\usepackage{upquote}}{}
% use microtype if available
\IfFileExists{microtype.sty}{%
\usepackage{microtype}
\UseMicrotypeSet[protrusion]{basicmath} % disable protrusion for tt fonts
}{}
\usepackage[margin=1in]{geometry}
\ifxetex
  \usepackage[setpagesize=false, % page size defined by xetex
              unicode=false, % unicode breaks when used with xetex
              xetex]{hyperref}
\else
  \usepackage[unicode=true]{hyperref}
\fi
\hypersetup{breaklinks=true,
            bookmarks=true,
            pdfauthor={Petr Keil},
            pdftitle={Math vs.~language, extinctions vs.~climate change},
            colorlinks=true,
            citecolor=blue,
            urlcolor=blue,
            linkcolor=magenta,
            pdfborder={0 0 0}}
\urlstyle{same}  % don't use monospace font for urls
\setlength{\parindent}{0pt}
\setlength{\parskip}{6pt plus 2pt minus 1pt}
\setlength{\emergencystretch}{3em}  % prevent overfull lines
\setcounter{secnumdepth}{0}

%%% Use protect on footnotes to avoid problems with footnotes in titles
\let\rmarkdownfootnote\footnote%
\def\footnote{\protect\rmarkdownfootnote}

%%% Change title format to be more compact
\usepackage{titling}
\setlength{\droptitle}{-2em}
  \title{Math vs.~language, extinctions vs.~climate change}
  \pretitle{\vspace{\droptitle}\centering\huge}
  \posttitle{\par}
  \author{Petr Keil}
  \preauthor{\centering\large\emph}
  \postauthor{\par}
  \predate{\centering\large\emph}
  \postdate{\par}
  \date{30/05/2015}




\begin{document}

\maketitle


Two unrelated insights that I've recently had:

\subsection{Math isn't more abstract than language, it's just
exact}\label{math-isnt-more-abstract-than-language-its-just-exact}

The reason is that mathematical notation is part of the same language
that we speak, and any mathematical formula is, in fact, a written (or
spoken) sentence. Example:

\[F(x)=\int_a^b \! f(x) \, \mathrm{d}x\]

Which is equivalent to: \emph{The value of function F at x equals to the
area of the region in the xy-plane that is bounded by the graph of f,
the x-axis, and the vertical lines at a and b.} Here, math is English.
But it can also be written as Chinese or Hungarian, or any other
language.

I realized that what makes math special in comparison with broader
language isn't its abstractness, but rather the opposite:
\textbf{mathematics emerges where things are exactly defined and follow
strict deductive logic}. The rest of language is a minefield of
metaphors, logical loopholes, and ambiguities.

Note: Of course language can be pretty abstract, and hence math can also
be very abstract. My insight concerns the relationship between the two.

\subsection{Extinctions and climate change are inverse scientific
problems}\label{extinctions-and-climate-change-are-inverse-scientific-problems}

\textbf{Climate change}: We know relatively well what has been
happenning with climate during the last 100 years -- we have all the
measurements. However, the link between the observed change and human
activity has been questioned.

\textbf{Species extinctions}: We know what happens when habitats shrink:
species go extinct and we lose biodiversity. This stems from the
relationship between area and the number of unique species in that area,
the \emph{Endemics-Area Relationship (EAR)}. We also know that habitable
area of the most diverse areas has been shrinking due to human activity.
However, we lack direct measurements of the actual loss of species --
not because they don't occur, but because an extinction is, when it
happens, difficult to register.

Putting them next to each other: We know that humans have been causing
extinctions, but we can't observe them. In contrast, we are not sure if
humans are causing climate change, but we can directly measure it. It's
inversed, in a way.

\end{document}
