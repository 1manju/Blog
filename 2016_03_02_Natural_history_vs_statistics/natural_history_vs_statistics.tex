\documentclass[11pt,a4paper]{article}
\usepackage[utf8]{inputenc}
\usepackage{amsmath}
\usepackage{amsfonts}
\usepackage{amssymb}
\usepackage{hyperref}

\title{Is natural history more fundamental than statistics?}
\author{Petr Keil}

\begin{document}

\maketitle

A couple of weeks ago at iDiv I had an exchange with Jonathan Chase about the importance of natural history, and whether it is more fundamental than statistics. Jon was arguing for fundamental importance of natural history, I disagreed. To quote Jon quoting Evelyn Hutchinson:

\begin{quote}
\textit{A quote Hutchinson wrote in 1975 about the importance of Natural history may shed some light on why I think that more fundamental than statistics, any good quantitative ecologist needs to be a good natural historian first…The quote is as relevant today (or more so) than when he wrote it 40 years ago:}
\begin{quote}
\textit{Modern biological education, however, may let us down as ecologists if it does not insist…that a wide and quite deep understanding of organisms, past and present, is as basic a requirement as anything else in ecological education.  It may be best self-taught, but how often is this difficult process made harder by a misplaced emphasis on a quite specious modernity.  Robert MacArthur really knew his warblers.}
\end{quote} 

\end{quote} 

I replied that dismissing statistics as specious modernity is unjust. Yes, there are bad statistical applications out there, and statistic can be used to conceal, divert attention, exaggerate, or to just make things too complex, but that is a problem of bad statisticians, not an inherent flaw of statistics. 

However, neither of us (including Hutchinson) provided good arguments for why natural history or statistics should really be that fundamental. We haven't continued the conversation, but the issue kept my brain busy nevertheless. Now I have this to say:

Any understanding of organisms begins with observation of nature or experimentation with it, or both. The observations and experiments then reveal rules (patterns, trends, laws) and exceptions to the rules, we are surprised or bored, we learn, we do more observations and experiments, we are surprised or bored, we learn, and so on. Children do it. People who do this professionally are scientists.

Now, in this process, where exactly is natural history and where is statistics? I'd say that \textbf{Natural history}, the deep understanding of nature, is the \textit{pile of rules and exceptions} that one accumulates during the cycle of observation, experimentation and learning. I'd also say that \textbf{Statistics} is \textit{the way to identify the rules and exceptions} among all the observations, and it does not really matter if this happens in children's brain, or in R.

From this perspective, it seems that both me and Jon were kind of right, because both statistics and natural history are really important; however, their relative importance is perhaps a matter of taste. But are they also fundamental (= foundational)? Or is there something even more fundamental that keeps them both alive?  Some infinite and renewable source of energy driving the cycle of observation, experimentation and learning?

I propose that such fundamental thing that keeps both natural history and statistics alive is curiosity. Simple curiosity. Maybe curiosity is the most fundamental basic skill that we, as scientists, should really profess and master (with a pinch of imagination and critical thinking). Deep knowledge and the ability to separate rules from exceptions will then come as a byproduct.
\\\\
PS: In a recent post, Brian McGill argues that \href{https://dynamicecology.wordpress.com/2016/02/29/we-arent-scientists-because-of-our-method-were-scientists-because-we-count/}{we are scientists because we count}. I think that it mixes nicely with my proposition that we are scientists because we are curious.
\\\\
PPS: Yesterday, Jon added that he did not want to question the relevance of statistics -- rather, he has been worried about the lack of hypotheses and actual thinking in many of the modern statistical analyses. I partly second that worry. However, I'd also be cautious here: hypothesis-free exploratory analysis does have its merits, if done right.

\end{document}